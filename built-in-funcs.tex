Operators such as \verb|+|, \verb|-|, etc. are used in infix form, e.g. 1 + 2 +
3. Functions are used in the familiar way: \verb|func(arg1, arg2)|.

We show the type signature of each function and operator. For example, {\tt *:
Real x RealMatrix -> RealMatrix} means that the operator \verb|*| takes a
\verb|Real| argument and a \verb|RealMatrix| argument, and produces a
\verb|RealMatrix| result.


\subsection{Constants}
\label{sec:builtin-constants}

\verb|null: Null| \reminder{?}

\verb|e: Real| The $e$ constant.

\verb|pi: Real| The $\pi$ constant.


\subsection{Comparison operators}
\label{sec:builtin-comparison}

\verb|<: Real x Real -> Boolean|

\verb|<=: Real x Real -> Boolean|

\verb|>: Real x Real -> Boolean|

\verb|>=: Real x Real -> Boolean|

\reminder{what about == and != ??? They're not in BuiltInFunctions.java}


\subsection{Arithmetic operations}
\label{sec:builtin-arithmetic}

\verb|+: Integer x Integer -> Integer| \\
\verb|+: Real x Real -> Real| \\
\verb|+: Timestep x Integer -> Timestep|

\verb|-: Integer x Integer -> Integer| \\
\verb|-: Real x Real -> Real| \\
\verb|-: Timestep x Integer -> Timestep|

\verb|*: Integer x Integer -> Integer| \\
\verb|*: Real x Real -> Real| \\
\verb|*: Timestep x Integer -> Timestep|

\verb|/: Integer x Integer -> Integer| \\
\verb|/: Real x Real -> Real| \\
\verb|/: Timestep x Integer -> Timestep|

\verb|^: Real x Real -> Real|

\verb|abs: Integer -> Integer| \\
\verb|abs: Real -> Real| Absolute value.

\verb|exp: Integer -> Real| \\
\verb|exp: Real -> Real| Exponential.

\reminder{no log??}

\verb|%: Integer x Integer -> Integer| \\
\verb|%: Timestep x Integer -> Timestep| Modulus.


\subsection{Matrix operations}
\label{sec:builtin-matrix-ops}

\verb|+: RealMatrix x RealMatrix -> RealMatrix|

\verb|-: RealMatrix x RealMatrix -> RealMatrix|

\verb|*: RealMatrix x RealMatrix -> RealMatrix| \\
\verb|*: RealMatrix x Real -> RealMatrix| \\
\verb|*: Real x RealMatrix -> RealMatrix|

\verb|inv: RealMatrix -> RealMatrix| Matrix inverse.

\verb|transpose: RealMatrix -> RealMatrix| Matrix transpose.

\verb|det: RealMatrix -> Real| Matrix determinant.

\verb|diag: RealMatrix -> RealMatrix| Takes a column vector of values, and
returns a diagonal matrix with the given values on the diagonal.

\verb|repmat: RealMatrix x Integer x Integer -> RealMatrix| Return a tiled
matrix. For example, \verb|repmat(x, 2, 3)| returns a matrix obtained by tiling
6 copies of \verb|x| in two rows and three columns.

\verb|sum: RealMatrix -> RealMatrix| Column-wise sum of a matrix. For example,
\verb|sum([1, 2; 3, 4])| returns \verb|[4, 6]|.

\verb|hstack: RealMatrix... -> RealMatrix| \\
\verb|hstack: Real... -> RealMatrix| Stacks scalars or matrices horizontally.
For example, \verb|hstack(1, 2, 3)| returns the row vector {\tt [1, 2, 3]}.

\verb|vstack: RealMatrix... -> RealMatrix| \\
\verb|vstack: Real... -> RealMatrix| Stacks scalars or matrices vertically. For
example, \verb|vstack(1, 2, 3)| returns the column vector {\tt [1; 2; 3]}.

\verb|eye: Integer -> RealMatrix| Returns an identity matrix of the given size.
For example, \verb|eye(5)| returns a 5x5 identity matrix.

\verb|zeros: Integer x Integer -> RealMatrix| Returns a matrix of the given size, filled with zeros.  For example, \verb|zeros(3, 4)| returns a 3x4 matrix of zeros.

\verb|ones: Integer x Integer -> RealMatrix| Returns a matrix of the given
size, filled with ones.  For example, \verb|ones(3, 4)| returns a 3x4 matrix of
ones.

\verb|abs: RealMatrix -> RealMatrix| Element-wise absolute value.

\verb|exp: RealMatrix -> RealMatrix| Element-wise exponential of a matrix.

\verb|[]: RealMatrix x Integer -> RealMatrix| Index a matrix. For example,
\verb|m[i]| returns the \verb|i|-th row of \verb|m|. \\
\verb|[]: Real[] x Integer -> Real| Index an array.

It is a runtime error to perform matrix operations with dimensions that do not
match.


\subsection{Set operations}
\label{sec:builtin-set-ops}

\verb|min: Set -> Integer| Minimum of a set.

\verb|max: Set -> Integer| Maximum of a set.

\verb|sum: Set -> Real| Sum of elements in a set of Real values.

\verb|size: Set -> Integer| Number of elements in a set.

\verb|iota: Set -> Any| Extract element from a singleton set. For example,
\verb|iota({6})| evaluates to \verb|6|.


\verb|round: Real -> Integer| Round a Real to the nearest Integer.  For
example, \verb|round(1.6)| evaluates to \verb|2|.

\verb|toInt: Real -> Integer| Convert to Integer, rounding towards zero. \\
\verb|toInt: Boolean -> Integer| Convert false to 0, true to 1. \\
\verb|toInt: Integer -> Integer| Identity. \\
\verb|toInt: RealMatrix -> Integer| Extract the element from a 1x1 RealMatrix,
and convert it to an Integer by rounding towards zero.

\verb|toReal: Real -> Real| Identity. \\
\verb|toReal: Boolean -> Real| Convert false to 0.0, true to 1.0. \\
\verb|toReal: Integer -> Real| Convert Integer to Real. \\
\verb|toReal: RealMatrix -> Real| Extract the element from a 1x1 RealMatrix,
and convert it to a Real.


\subsection{Trigonometric functions}
\label{sec:builtin-trig}

\verb|sin: Real -> Real| Sine of the given value in radians.

\verb|cos: Real -> Real| Cosine of the given value in radians.

\verb|tan: Real -> Real| Tangent of the given value in radians.

\verb|atan2: Real x Real -> Real| Arctangent. Analogous to the \verb|atan2|
function in Java.


\subsection{Miscellaneous functions}
\label{sec:builtin-misc}

\verb|succ: Integer -> Integer| Return the next integer.

\verb|pred: Integer -> Integer| Return the previous integer.

\verb|next: Timestep -> Timestep| Return the next timestep.

\verb|prev: Timestep -> Timestep| Return the next timestep.

\verb|+: String x String -> String| Concatenate strings.

\verb|isEmptyString: String -> Boolean| Return true iff string is empty.

\verb|loadRealMatrix: String -> RealMatrix| Load a \verb|RealMatrix| from a
text file. The space-separated formats produced by numpy and Matlab are
supported.


