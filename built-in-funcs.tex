\subsection{Unary operators}
\label{sec:builtin-unary-ops}

The operators are listed in order of precedence, with highest precedence at the
top.

% NOTE: Check BLOGParser.cup for precedence.
\begin{table}[H]
\centering
\caption{Unitary Operators}
\begin{tabular}{ c c c c }
\toprule
operator & argument type & result type & meaning \\
\midrule
\verb|-| & Integer & Integer & minus \\
\verb|-| & Real & Real & minus \\
\verb|-| & RealMatrix & RealMatrix & minus \\
\verb|!| & Boolean & Boolean & negation \\
\bottomrule
\end{tabular}
\end{table}



\subsection{Binary operators}
\label{sec:builtin-binary-ops}

The operators are listed in order of precedence, with highest precedence at the
top.

% NOTE: Check BLOGParser.cup for precedence.
\begin{table}[H]
\centering
\caption{Binary Operators}
\begin{tabular}{ c c c c c }
\toprule
left-hand type & operator & right-hand type & result type & meaning \\
\midrule

Integer & \verb|*| & Integer & Integer & multiply \\
Real & \verb|*| & Real & Real & multiply \\
Timestep & \verb|*| & Integer & Timestep & multiply \\

Integer & \verb|/| & Integer & Integer & divide \\
Real & \verb|/| & Real & Real & divide \\
Timestep & \verb|/| & Integer & Timestep & divide \\

Integer & \verb|%| & Integer & Integer & modulus \\

Real & \verb|^| & Real & Real & power \\

Integer & \verb|+| & Integer & Integer & plus \\
Real & \verb|+| & Real & Real & plus \\
Timestep & \verb|+| & Integer & Timestep & plus \\
String & \verb|+| & String & String & concatenate \\

Integer & \verb|-| & Integer & Integer & minus \\
Real & \verb|-| & Real & Real & minus \\
Timestep & \verb|-| & Integer & Timestep & minus \\

Integer & \verb|<| & Integer & Boolean & less than \\
Real & \verb|<| & Real & Boolean & less than \\
Timestep & \verb|<| & Timestep & Boolean & less than \\

Integer & \verb|>| & Integer & Boolean & greater than \\
Real & \verb|>| & Real & Boolean & greater than \\
Timestep & \verb|>| & Timestep & Boolean & greater than \\

Integer & \verb|<=| & Integer & Boolean & less than or equal \\
Real & \verb|<=| & Real & Boolean & less than or equal \\
Timestep & \verb|<=| & Timestep & Boolean & less than or equal \\

Integer & \verb|>=| & Integer & Boolean & greater than or equal \\
Real & \verb|>=| & Real & Boolean & greater than or equal \\
Timestep & \verb|>=| & Timestep & Boolean & greater than or equal \\

Any & \verb|==| & Any & Boolean & equal to \\

Any & \verb|!=| & Any & Boolean & not equal to \\

Boolean & \verb|&| & Boolean & Boolean & and \\

Boolean & \verb||| & Boolean & Boolean & or \\

Boolean & \verb|=>| & Boolean & Boolean & implies \\

\bottomrule
\end{tabular}
\end{table}


\subsection{Constants}
\label{sec:builtin-constants}

\blog|Real e|
\myindent The base of the natural logarithm (approximately equal to $2.71828$).

\blog|Real pi|
\myindent The $\pi$ constant (approximately equal to $3.14159$).


\subsection{Matrix operations}
\label{sec:builtin-matrix-ops}

For functions, a type signature like \verb|RealMatrix eye(Integer dim)| means
that the function \verb|eye| takes a single \verb|Integer| and produces a
\verb|RealMatrix|.

\blog|RealMatrix inv(RealMatrix x)|
\myindent Inverse of the matrix \verb|x|.

\blog|RealMatrix transpose(RealMatrix x)|
\myindent Transpose of the matrix \verb|x|.

\blog|RealMatrix det(RealMatrix x)|
\myindent Determinant of the matrix \verb|x|.

\blog|RealMatrix diag(RealMatrix vals)|
\myindent Diagonal matrix with the given values on the diagonal. \verb|vals|
must be a
column vector.

\blog|RealMatrix repmat(RealMatrix m, Integer rows, Integer cols)|
\myindent Return the matrix \verb|m| tiled \verb|rows| times vertically and
\verb|cols|
times horizontally.

\blog|RealMatrix sum(RealMatrix m)|
\myindent Column-wise sum of a matrix.
For example, \verb|sum([1, 2; 3, 4])| returns \verb|[4, 6]|.

\blog|RealMatrix hstack(RealMatrix arg1, ...)|
\blog|RealMatrix hstack(Real arg1, ...)|
\myindent Stack scalars or matrices horizontally. Accepts an arbitrary number
of
arguments.
For example, \verb|hstack(1, 2, 3)| returns the row vector {\tt [1, 2, 3]}.

\blog|RealMatrix vstack(RealMatrix arg1, ...)|
\blog|RealMatrix vstack(Real arg1, ...)|
\myindent Stack scalars or matrices vertically. Accepts an arbitrary number of
arguments.
For example, \verb|vstack(1, 2, 3)| returns the column vector {\tt [1; 2; 3]}.

\blog|RealMatrix eye(Integer dim)|
\myindent Identity matrix of the given size.
For example, \verb|eye(5)| returns a 5x5 identity matrix.

\blog|RealMatrix zeros(Integer rows, Integer cols)|
\myindent Matrix of the given size, filled with zeros.
For example, \verb|zeros(3, 4)| returns a 3x4 matrix of zeros.

\blog|RealMatrix ones(Integer rows, Integer cols)|
\myindent Matrix of the given size, filled with ones.
For example, \verb|ones(3, 4)| returns a 3x4 matrix of ones.

\blog|RealMatrix abs(RealMatrix m)|
\myindent Element-wise absolute value.

\blog|RealMatrix exp(RealMatrix m)|
\myindent Element-wise exponential of a matrix.

Use square brackets to index into a matrix. If \verb|m| is a two-dimensional
matrix, \verb|m[i]| returns the \verb|i|-th row of \verb|m|. If \verb|m| is a
row or column vector, \verb|m[i]| returns the \verb|i|-th element of \verb|m|.
Note that the resulting value is a \verb|RealMatrix|. To get the i,j-th
element, use \verb|toInt(m[i][j])|.

It is a runtime error to perform matrix operations with dimensions that do not
match.


\subsection{Set operations}
\label{sec:builtin-set-ops}

\blog|Any min(Set s)|
\myindent Minimum of a set.

\blog|Any max(Set s)|
\myindent Maximum of a set.

\blog|Real sum(Set s)|
\myindent Sum of elements in a set of Real values.

\blog|Integer size(Set s)|
\myindent Number of elements in a set.

\blog|Any iota(Set s)|
\myindent Extract element from a singleton set.
For example, \verb|iota({6})| evaluates to \verb|6|.

Note: \verb|Any| is an internal BLOG type not meant to be used by the end user.
It is necessary here because the BLOG type system does not distinguish between,
for example, a \verb|Set| of \verb|Real|s and a \verb|Set| of \verb|Integer|s.
Assigning an \verb|Any| value to a variable of the wrong type is a runtime
error.


\subsection{Conversions between types}
\label{sec:builtin-conversions}

\blog|Integer round(Real val)|
\myindent Round a Real to the nearest Integer.
For example, \verb|round(1.6)| evaluates to \verb|2|.

\blog|Integer toInt(Real val)|
\myindent Convert to Integer, rounding towards zero.

\blog|Integer toInt(Boolean val)|
\myindent Convert false to 0, true to 1.

\blog|Integer toInt(RealMatrix val)|
\myindent Extract the element from a 1x1 RealMatrix, and convert it to an
Integer by rounding towards zero.

\blog|Real toReal(Boolean val)|
\myindent Convert false to 0.0, true to 1.0.

\blog|Real toReal(Integer val)|
\myindent Convert Integer to Real.

\blog|Real toReal(RealMatrix val)|
\myindent Extract the element from a 1x1 RealMatrix, and convert it to a Real.


\subsection{Trigonometric functions}
\label{sec:builtin-trig}

\blog|Real sin(Real radians)|
\myindent Sine of the given value.

\blog|Real cos(Real radians)|
\myindent Cosine of the given value.

\blog|Real tan(Real radians)|
\myindent Tangent of the given value.

\blog|Real atan2(Real y, Real x)|
\myindent Arctangent. Analogous to the \verb|atan2| function in Java.


\subsection{Miscellaneous functions}
\label{sec:builtin-misc}

\blog|Integer abs(Integer x)|
\blog|Real abs(Real x)|
\myindent Absolute value.

\blog|Real exp(Integer x)|
\blog|Real exp(Real x)|
\myindent Exponential.

\blog|Real log(Integer x)|
\blog|Real log(Real x)|
\myindent Natural logarithm.

\blog|Timestep next(Timestep t)|
\myindent The next timestep.

\blog|Timestep prev(Timestep t)|
\myindent The previous timestep.

\blog|Boolean isEmptyString(String s)|
\myindent True iff string is empty.

\blog|RealMatrix loadRealMatrix(String s)|
\myindent Load a \verb|RealMatrix| from a text file. The space-separated
formats produced by numpy and Matlab are supported.
