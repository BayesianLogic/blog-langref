Operators such as \verb|+|, \verb|-|, etc. are used in infix form, e.g. {\tt 1
+ 2 - 3}. A type signature like \verb|(Integer) Integer + Integer| means that
the operator \verb|+| takes two \verb|Integer|s and produces an \verb|Integer|.
For functions, a type signature like \verb|RealMatrix eye(Integer dim)| means
that the function \verb|eye| takes a single \verb|Integer| and produces a
\verb|RealMatrix|.


\subsection{Constants}
\label{sec:builtin-constants}

\verb|Real e| \\
The base of the natural logarithm (approximately equal to $2.71828$).

\verb|Real pi| \\
The $\pi$ constant (approximately equal to $3.14159$).


\subsection{Comparison operators}
\label{sec:builtin-comparison}

\verb|(Boolean) Integer < Integer| \\
\verb|(Boolean) Real < Real|

\verb|(Boolean) Integer <= Integer| \\
\verb|(Boolean) Real <= Real|

\verb|(Boolean) Integer > Integer| \\
\verb|(Boolean) Real > Real|

\verb|(Boolean) Integer >= Integer| \\
\verb|(Boolean) Real >= Real|

\verb|(Boolean) Any == Any|

\verb|(Boolean) Any != Any|


\subsection{Arithmetic operations}
\label{sec:builtin-arithmetic}

\verb|(Integer) Integer + Integer| \\
\verb|(Real) Real + Real| \\
\verb|(Timestep) Timestep + Integer|

\verb|(Integer) Integer - Integer| \\
\verb|(Real) Real - Real| \\
\verb|(Timestep) Timestep - Integer|

\verb|(Integer) Integer * Integer| \\
\verb|(Real) Real * Real| \\
\verb|(Timestep) Timestep * Integer|

\verb|(Integer) Integer / Integer| \\
\verb|(Real) Real / Real| \\
\verb|(Timestep) Timestep / Integer|

\verb|(Real) Real ^ Real| \\
Raising to a power.

\verb|(Integer) Integer % Integer| \\
\verb|(Timestep) Timestep % Integer| \\
Modulus.

\verb|Integer abs(Integer x)| \\
\verb|Real abs(Real x)| \\
Absolute value.

\verb|Real exp(Integer x)| \\
\verb|Real exp(Real x)| \\
Exponential.

\verb|Real log(Integer x)| \\
\verb|Real log(Real x)| \\
Natural logarithm.


\subsection{Matrix operations}
\label{sec:builtin-matrix-ops}

\verb|(RealMatrix) RealMatrix + RealMatrix|

\verb|(RealMatrix) RealMatrix - RealMatrix|

\verb|(RealMatrix) RealMatrix * RealMatrix| \\
\verb|(RealMatrix) RealMatrix * Real| \\
\verb|(RealMatrix) Real * RealMatrix|

\verb|RealMatrix inv(RealMatrix x)| \\
Inverse of the matrix \verb|x|.

\verb|RealMatrix transpose(RealMatrix x)| \\
Transpose of the matrix \verb|x|.

\verb|RealMatrix det(RealMatrix x)| \\
Determinant of the matrix \verb|x|.

\verb|RealMatrix diag(RealMatrix vals)| \\
Diagonal matrix with the given values on the diagonal. \verb|vals| must be a
column vector.

\verb|RealMatrix repmat(RealMatrix m, Integer rows, Integer cols)| \\
Return the matrix \verb|m| tiled \verb|rows| times vertically and \verb|cols|
times horizontally.

\verb|RealMatrix sum(RealMatrix m)| \\
Column-wise sum of a matrix.
For example, \verb|sum([1, 2; 3, 4])| returns \verb|[4, 6]|.

\verb|RealMatrix hstack(RealMatrix, ...)| \\
\verb|RealMatrix hstack(Real, ...)| \\
Stack scalars or matrices horizontally. Accepts an arbitrary number of
arguments.
For example, \verb|hstack(1, 2, 3)| returns the row vector {\tt [1, 2, 3]}.

\verb|RealMatrix vstack(RealMatrix, ...)| \\
\verb|RealMatrix vstack(Real, ...)| \\
Stack scalars or matrices vertically. Accepts an arbitrary number of arguments.
For example, \verb|vstack(1, 2, 3)| returns the column vector {\tt [1; 2; 3]}.

\verb|RealMatrix eye(Integer dim)| \\
Identity matrix of the given size.
For example, \verb|eye(5)| returns a 5x5 identity matrix.

\verb|RealMatrix zeros(Integer rows, Integer cols)| \\
Matrix of the given size, filled with zeros.
For example, \verb|zeros(3, 4)| returns a 3x4 matrix of zeros.

\verb|RealMatrix ones(Integer rows, Integer cols)| \\
Matrix of the given size, filled with ones.
For example, \verb|ones(3, 4)| returns a 3x4 matrix of ones.

\verb|RealMatrix abs(RealMatrix m)| \\
Element-wise absolute value.

\verb|RealMatrix exp(RealMatrix m)| \\
Element-wise exponential of a matrix.

\verb|(RealMatrix) RealMatrix[Integer]| \\
Index into a matrix. If \verb|m| is a two-dimensional matrix, \verb|m[i]|
returns the \verb|i|-th row of \verb|m|. If \verb|m| is a row or column vector,
\verb|m[i]| returns the \verb|i|-th element of \verb|m|. Note that the
resulting value is a \verb|RealMatrix|. To get the i,j-th element, use
\verb|toInt(m[i][j])|.

It is a runtime error to perform matrix operations with dimensions that do not
match.


\subsection{Set operations}
\label{sec:builtin-set-ops}

\verb|Any min(Set s)| \\
Minimum of a set.

\verb|Any max(Set s)| \\
Maximum of a set.

\verb|Real sum(Set s)| \\
Sum of elements in a set of Real values.

\verb|Integer size(Set s)| \\
Number of elements in a set.

\verb|Any iota(Set s)| \\
Extract element from a singleton set.
For example, \verb|iota({6})| evaluates to \verb|6|.


\subsection{Conversions between types}
\label{sec:builtin-conversions}

\verb|Integer round(Real val)| \\
Round a Real to the nearest Integer.
For example, \verb|round(1.6)| evaluates to \verb|2|.

\verb|Integer toInt(Real val)| \\
Convert to Integer, rounding towards zero.

\verb|Integer toInt(Boolean val)| \\
Convert false to 0, true to 1.

\verb|Integer toInt(RealMatrix val)| \\
Extract the element from a 1x1 RealMatrix, and convert it to an Integer by
rounding towards zero.

\verb|Real toReal(Boolean val)| \\
Convert false to 0.0, true to 1.0.

\verb|Real toReal(Integer val)| \\
Convert Integer to Real.

\verb|Real toReal(RealMatrix val)| \\
Extract the element from a 1x1 RealMatrix, and convert it to a Real.


\subsection{Trigonometric functions}
\label{sec:builtin-trig}

\verb|Real sin(Real radians)| \\
Sine of the given value.

\verb|Real cos(Real radians)| \\
Cosine of the given value.

\verb|Real tan(Real radians)| \\
Tangent of the given value.

\verb|Real atan2(Real y, Real x)| \\
Arctangent. Analogous to the \verb|atan2| function in Java.


\subsection{Miscellaneous functions}
\label{sec:builtin-misc}

\verb|Integer succ(Integer n)| \\
The next integer.

\verb|Integer pred(Integer n)| \\
The previous integer.

\verb|Timestep next(Timestep t)| \\
The next timestep.

\verb|Timestep prev(Timestep t)| \\
The previous timestep.

\verb|(String) String + String| \\
Concatenate strings.

\verb|Boolean isEmptyString(String s)| \\
True iff string is empty.

\verb|RealMatrix loadRealMatrix(String s)| \\
Load a \verb|RealMatrix| from a text file. The space-separated formats produced
by numpy and Matlab are supported.
