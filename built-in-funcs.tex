\subsection{Scalar operations}
\label{sec:builtin-scalar-ops}

The following operators are defined on Integer and Real scalars:

\begin{table}[H]
\centering
\begin{tabular}{ c c c }
\toprule 
operator & interpretation & example \\
\midrule
\verb|+| & plus & \verb|x + y| , \verb|1.0 + 2|\\ 
\verb|-| & minus & \verb|x - y| , \verb|1.0 - 2|\\ 
\verb|*| & multiply & \verb|x * y| , \verb|1.0 * 2|\\
\verb|/| & divide & \verb|x / y| , \verb|1.0 / 2|\\ 
\verb|%| & modulus (only applied to Integers) & \verb|x % y|, \verb|1.0 % 2| \\
\verb|^| & power & \verb|x ^ y| , \verb|1.0 ^ 2| \\
\bottomrule
\end{tabular}
\end{table}

The following functions are defined on Integer and Real scalars:

\begin{itemize}
\item
    \verb|abs| computes the absolute value of a number. \\
    For example, \verb|abs(-1.2)| evaluates to \verb|1.2|.
\item
    \verb|exp| computes the exponential of a number. \\
    For example, \verb|exp(2.5)| evaluates to \verb|12.1825|.
\item
    \verb|round| rounds a Real to the nearest Integer. \\
    For example, \verb|round(1.6)| evaluates to \verb|2|.
\end{itemize}


\subsection{Matrix operations}
\label{sec:builtin-matrix-ops}

RealMatrix represents a 2d matrix. Column vectors and row vectors are just
special cases where the number of rows or columns is one. BLOG does not
distinguish between vectors and matrices at the type level. The following
operators are defined for the RealMatrix type:

\begin{table}[H]
\centering
\begin{tabular}{ c c c }
\toprule 
operator & interpretation & example \\
\midrule
\verb|+| & plus & \verb|x + y| \\ 
\verb|-| & minus & \verb|x - y| \\ 
\verb|*| & multiply & \verb|x * y| \\
\bottomrule
\end{tabular}
\end{table}

The following functions are defined for the RealMatrix type:

\begin{itemize}
\item
    \verb|inv| computes the inverse of a matrix.
\item
    \verb|transpose| computes the transpose of a matrix.
\item
    \verb|det| computes the determinant of a matrix.
\item
    \verb|repmat| returns a tiled matrix. \\
    For example, \verb|repmat(x, 2, 3)| returns a matrix obtained by tiling 6
    copies of \verb|x| in two rows and three columns.
\item
    \verb|diag| takes a column vector of values, and returns a diagonal matrix
    with the given values on the diagonal.
\item
    \verb|vstack| stacks scalars or matrices vertically. \\
    For example, \verb|vstack(1, 2, 3)| returns the column vector {\tt [1; 2;
    3]}.
\item
    \verb|hstack| stacks scalars or matrices horizontally. \\
    For example, \verb|hstack(1, 2, 3)| returns the row vector {\tt [1, 2, 3]}.
\item
    \verb|eye| returns an identity matrix of the given size. \\
    For example, \verb|eye(5)| returns a 5x5 identity matrix.
\item
    \verb|zeros| returns a matrix of the given size, filled with zeros. \\
    For example, \verb|zeros(3, 4)| returns a 3x4 matrix of zeros.
\item
    \verb|ones| returns a matrix of the given size, filled with ones. \\
    For example, \verb|ones(3, 4)| returns a 3x4 matrix of ones.
\item
    \verb|exp| returns the element-wise exponential of a matrix. \\
\end{itemize}

It is a runtime error to perform matrix operations with dimensions that do not
match.


\subsection{Conversions between types}
\label{sec:builtin-conversions}

The following functions allow converting between types:

\begin{itemize}
\item
    \verb|toReal| takes a 1x1 RealMatrix, an Integer, or a Boolean, and
    converts it to a Real. \\
    For example, if \verb|x| is a matrix, then \verb|x[0][0]| is a 1x1
    RealMatrix. To extract its Real value, use \verb|toReal(x[0][0])|.
    For Booleans, \verb|toReal(true)| evaluates to \verb|1| and
    \verb|toReal(false)| evaluates to \verb|0|.
\item
    \verb|toInt| takes a 1x1 RealMatrix, a Real, or a Boolean, and converts it
    to an Integer. \\
    For example, \verb|toInt(2.9)| evaluates to \verb|2|. (The decimal part is
    dropped.)
\end{itemize}


\reminder{stuff below is still to do}

\begin{table}[H]
\centering
\caption{Logical operators on Boolean}
\begin{tabular}{ c c c }
\toprule 
operator & interpretation & example \\
\midrule
\verb|&| & and & \verb|x & y| , \verb|(x > 3) & (x < 5)| \\ 
{\tt |} & or & \verb#x | y# , \verb#(x > 5) | (x < 3)# \\ 
{\tt !} & not & \verb|! x| , \verb|! (x > 3)| \\
{\tt =>} & imply & \verb|x => y| , \verb|(x > 5) => (x > 3)| \\
\bottomrule
\end{tabular}
\end{table}

\begin{table}[H]
\centering
\caption{Quantified formula}
\begin{tabular}{ c c c }
\toprule 
operator & interpretation & example \\
\midrule
{\tt forall} & $\forall$ & \verb|forall Person x height(x) > 1.0| \\ 
{\tt exists} & $\exists$ & \verb|exists Person x height(x) > 1.0|  \\ 
\bottomrule
\end{tabular}
\end{table}

\begin{table}[H]
\centering
\caption{Relational operators on Integer, Real and other comparable types}
\begin{tabular}{ c c c }
\toprule 
operator & interpretation & example \\
\midrule
{\tt >} & greater than & \verb|a > b| ,  \verb|2 > 1.0|\\ 
{\tt >=} & greater than or equal to & \verb|a >= b| ,  \verb|2 >= 1.0| \\ 
{\tt <} & less than & \verb|a < b| ,  \verb|1.0 < 2.0| \\
{\tt <=} & less than or equal to & \verb|a <= b| ,  \verb|1.0 <= 2.0|  \\
\bottomrule
\end{tabular}
\end{table}

\begin{table}[H]
\centering
\caption{Equality operator on all types}
\begin{tabular}{ c c c }
\toprule 
operator & interpretation & example \\
\midrule
{\tt ==} & equal to & \verb|a == b|   \\ 
{\tt !=} & unequal to & \verb|a != b|  \\ 
\bottomrule
\end{tabular}
\end{table}

\begin{table}[H]
\centering
\caption{Operators on String}
\begin{tabular}{ c c c }
\toprule 
operator & interpretation & example \\
\midrule
{\tt +} & concatenate & \verb|"hello " + "world"|  \\ 
{\tt ==} & equal to & \verb|"abc" == "def"|  \\
{\tt !=} & unequal to & \verb|"abc" != "def"|\\ 
\verb|IsEmptyString()| & returns True if the string is empty & \verb|IsEmptyString(a)| \\
\bottomrule
\end{tabular}
\end{table}

\begin{table}[H]
\centering
\caption{Operators on Timestep}
\begin{tabular}{ c c c }
\toprule 
operator & interpretation & example \\
\midrule
\verb|prev()| & previous Timestep & \verb|prev(@1)|  \\
\verb|-| & Timestep minus an integer & \verb|@10 - 1 == @9|  \\
\verb|+| & Timestep plus an integer & \verb|@10 + 1 == @11|  \\
\verb|%| &  Timestep mod & \verb|x % 10 == @0|  \\
\verb|*| & Timestep multiply an integer & \verb|@10 * 2 == @20|  \\
\verb|/| & Timestep divide an integer & \verb|@10 / 2 == @5|  \\
\bottomrule
\end{tabular}
\end{table}

\begin{table}[H]
\centering
\caption{Arithmetic operators on Set}
\begin{tabular}{ c c c }
\toprule 
operator & interpretation & example \\
\midrule
\verb|min| & minimum of elements a set & \verb|min()|\\ 
\verb|max| & maximum of elements in a set & \verb|max()|\\
\verb|sum| & summation of elements in a set & \verb|sum()|\\ 
\bottomrule
\end{tabular}
\end{table}

