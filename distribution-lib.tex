This section lists the distributions included in \bl standard library.

\subsection{Bernoulli($p$)}
Bernoulli distribution generates the value $1$ with probability $p$ and $0$ with probability $1-p$. 

\paragraph*{Parameters:} 
\begin{itemize*}
\item[] \verb|Real|
 $p$, $0 \leq p \leq 1$ 
\end{itemize*}
\paragraph*{Support:} \verb|Integer|, $k \in \{0, 1\}$ 

\paragraph*{Probability mass function:}
\[
	P(X = k) = \left\{
	  \begin{array}{lr}
	    p &    k = 1 \\
	    1-p &  k = 0
	  \end{array}
	\right.
\]

\paragraph*{Example:}
The following code defines a random function symbol \verb|x| distributed according to a Bernoulli distribution.
\begin{blogcode}
random Integer x ~ Bernoulli(0.5);
\end{blogcode}

\subsection{Beta($\alpha$, $\beta$)}

\paragraph*{Parameters:} 
\begin{itemize*}
\item[] \verb|Real| $\alpha$, $\alpha > 0$ 
\item[] \verb|Real| $\beta$, $\beta > 0$ 
\end{itemize*}

\paragraph*{Support:} \verb|Real|, $0 \leq x \leq 1$ 

\paragraph*{Probability density function:}
\[
	f(x) = \frac{x^{\alpha - 1} (1-x)^{\beta - 1}}{\int_{0}^{1} x^{\alpha - 1} (1-x)^{\beta - 1} dx}
\]

\paragraph*{Example:}
The following code defines a random function symbol \verb|x| distributed according to a Beta distribution with $\alpha = 2$ and $\beta = 3$
\begin{blogcode}
random Real x ~ Beta(2, 3);
\end{blogcode}


\subsection{Binomial($n$, $p$)} 
Binomial distribution generates the number of successes in a sequence of $n$ independent Bernoulli trials where each trial yields success with probability $p$.

\paragraph*{Parameters:} 

\begin{itemize*}
\item[] \verb|Integer| $n$, $n \geq 0$
\item[] \verb|Real| $p$, $0 \leq p \leq 1$
\end{itemize*}

\paragraph*{Support:} \verb|Integer|, $k \in \{0, 1, \ldots, n\} $

\paragraph*{Probability mass function:}
\[
	P(X = k) = \binom{n}{k}p^{k}(1-p)^{k} , 0 \leq k \leq n \\
\]

\paragraph*{Example:}
The following code defines a random function symbol \verb|x| distributed according to a Binomial distribution.
\begin{blogcode}
random Integer x ~ Binomial(10, 0.5);
\end{blogcode}

\subsection{BooleanDistrib($p$)}
BooleanDistrib distribution generates \verb|true| with probability $p$ and \verb|false| with probability $1-p$. 

\paragraph*{Parameters:} 
\begin{itemize*}
\item[] \verb|Real|
 $p$, $0 \leq p \leq 1$ 
\end{itemize*}
\paragraph*{Support:} \verb|Boolean|, $k \in \{\verb|True|, \verb|False|\}$ 

\paragraph*{Probability mass function:}
\[
	P(X = k) = \left\{
	  \begin{array}{lr}
	    p & k = \verb|true| \\
	    1 - p & k = \verb|false|
	  \end{array}
	\right.
\]

\paragraph*{Example:}
The following code defines a random function symbol \verb|x| distributed according to a BooleanDistrib distribution.
\begin{blogcode}
random Boolean x ~ BooleanDistrib(0.5);
\end{blogcode}

\subsection{Categorical}
Refer to Section 9.1.12

\subsection{Dirichlet($\bar{\alpha}$)}

\paragraph*{Parameters:} 
\begin{itemize*}
\item[] \verb|RealMatrix| (Column Vector) $\bar{\alpha}$, s.t. $\alpha_{i} \geq 0, \forall i \in \{1, \ldots, K\}$ 
\end{itemize*}

\paragraph*{Support:} 
\[
\verb|RealMatrix|, \bar{x} \in \mathbb{R}^{K}, s.t. \forall i \in \{1, \ldots, K\} x_{i} \geq 0, \sum_{1}^{K} x_{i} = 1 
\]

\paragraph*{Probability density function:}
\[
	f(\bar{x}) = \frac{1}{Z(\bar{\alpha})} * \prod_{i=1}^{K} x_{i}^{\alpha_{i} - 1}
\]
\[ Z(\bar{\alpha}) = \frac{ \prod_{i=1}^{K} \Gamma(\alpha_{i})   }{ \Gamma(\sum_{i=1}^{K} \alpha_{i} )  } 
\]

\paragraph*{Example:}
The following code defines a random function symbol \verb|x| distributed according to a Dirichlet distribution with $\bar{\alpha} = (1, 3, 5) $
\begin{blogcode}
fixed RealMatrix vector = [1; 3; 5];
random RealMatrix x ~ Dirichlet(matrix);
\end{blogcode}

\subsection{Discrete($\bar{p}$)}
The Discrete distribution generates an integer from $0$ to $K-1$, where the likelihood of each integer is represented by the unnormalized probability vector $\bar{p}$. The Discrete distribution is a simplification of the Multinomial Distribution where $n=1$.

\paragraph*{Parameters:}
\begin{itemize*}
\item[] \verb|RealMatrix| (ColumnVector) $\bar{p}$, $\bar{p} \in \mathbb{R}$, $p_{0} \geq 0, \ldots, p_{K-1} \geq 0, s.t. \sum_{0}^{K-1} p_{i} > 0$
\end{itemize*}

\paragraph*{Support:} \verb|Integer| $k \in \{0, 1, \ldots, K-1 \}$ 

\paragraph*{Probability mass function:}
\[
	P(X = k) = \frac{p_{k}}{\sum_{i=0}^{K -1} p_{i}}
\]

\paragraph*{Example:}
The following code defines a random function symbol \verb|x| distributed according to a Discrete distribution with normalized probabilities $0.5, 0.3, 0.2$.
\begin{blogcode}
fixed RealMatrix vec = [5; 3; 2];
random Integer x ~ Discrete(vec);
\end{blogcode}

\subsection{Exponential($\lambda$)} 

\paragraph*{Parameters:} 
\begin{itemize*}
\item[] \verb|Real|
 $\lambda$, $\lambda > 0$ 
\end{itemize*}
\paragraph*{Support:} \verb|Real|, $x \geq 0$ 

\paragraph*{Probability density function:}
\[
	f(x) = \lambda \exp(-\lambda x) , x \geq 0 \\
\]

\paragraph*{Example:}
The following code defines a random function symbol \verb|x| distributed according to an Exponential distribution.
\begin{blogcode}
random Real x ~ Exponential(0.2);
\end{blogcode}

\subsection{Gamma($k$,$\lambda$)}


\subsection{Gaussian($\mu$,$\sigma$)}

\paragraph*{Parameters:} 
\begin{itemize*}
\item[] \verb|Real|
 $\mu$, $\mu \in \mathbb{R}$
\item[] \verb|Real|
 $\sigma$, $\sigma \geq 0$ 
\end{itemize*}

\paragraph*{Support:} \verb|Real|, $x \in \mathbb{R}$ 

\paragraph*{Probability density function:}
\[
	f(x) = \frac{1}{\sigma \sqrt{2 \pi}} exp(\frac{-(x-\mu)^{2}}{2 \sigma^{2}})
\]

\paragraph*{Example:}
The following code defines a random function symbol \verb|x| distributed according to a Gaussian distribution with mean $-1$ and variance $2$.
\begin{blogcode}
random Real x ~ Gaussian(-1, 2);
\end{blogcode}

\subsection{Geometric($p$)}
Geometric distribution generates the number of failures before the first success in a sequence of independent Bernoulli trials with probability of success $p$.

\paragraph*{Parameters:} 
\begin{itemize*}
\item[] \verb|Real|
 $p$, $0 \leq p \leq 1$ 
\end{itemize*}
\paragraph*{Support:} \verb|Integer|, $k \in \{0, 1, 2, \ldots \}$ 

\paragraph*{Probability mass function:}
\[
	P(X = k) = p(1-p)^{k}
\]

\paragraph*{Example:}
The following code defines a random function symbol \verb|x| distributed according to a Geometric distribution where each Bernoulli trial has a probability of success $p = 0.2$
\begin{blogcode}
random Integer x ~ Geometric(0.2);
\end{blogcode}

\subsection{Laplace($\mu$,$b$)}

\paragraph*{Parameters:} 
\begin{itemize*}
\item[] \verb|Real| $\mu$, $\mu \in \mathbb{R}$
\item[] \verb|Real| $b$, $b > 0$
 
\end{itemize*}
\paragraph*{Support:} \verb|Real|, $x \in \mathbb{R}$ 

\paragraph*{Probability mass function:}
\[
	f(x) = \frac{1}{2b} exp(\frac{-|x-\mu|}{b})
\]

\paragraph*{Example:}
The following code defines a random function symbol \verb|x| distributed according to a Laplace distribution with mean $\mu = 1.0$ and scale $b = 2.0$
\begin{blogcode}
random Real x ~ Laplace(1.0, 2.0);
\end{blogcode}


\subsection{Multinomial($n$, $\bar{p}$)}
 The multinomial distribution contains $k$ categories. During a single trial, exactly one of the $k$ categories is selected. The probability of selecting a particular category $i$ is $\frac{p_{i}}{\sum_{i=1}^{k} p_{i}}$. The process is repeated independently $n$ times, and the resultant counts of each category are returned.
 
\paragraph*{Parameters:} 
\begin{itemize*}
\item[] \verb|Integer| $n$, $n \in \{0, 1, 2, \ldots \}$
\item[] \verb|RealMatrix| (Column Vector) $\bar{p}$, $p \in \mathbb{R}^{k}$ $p_{1} \geq 0, p_{2} \geq 0, \ldots, .. p_{k} \geq 0, s.t. \sum_{i=1}^{k} p_{i} > 0$
\end{itemize*}

\paragraph*{Support:} \verb|RealMatrix| (Column Vector), $\bar{x} \in \mathbb{Z}^{k}, s.t. x_{i} \in \{0, 1, 2, \ldots, n \} \forall i \in \{0, 1, 2, \ldots, k \}  $ 

\paragraph*{Probability mass function:}
\[
	P(X = x_{1},\ldots,x_{k}) = \frac{n!}{x_{1}! \times \ldots \times x_{k}!} p_{1}^{x_{1}} \ldots p_{k}^{x_{k}}
\]

\paragraph*{Example:}
The following code defines a random function symbol \verb|x| distributed according to a Multinomial distribution.
\begin{blogcode}
fixed RealMatrix m = [1; 1; 2];
random RealMatrix x ~ Multinomial(4, m);
\end{blogcode}

\subsection{MultivarGaussian($\bar{\mu}$, $\Sigma$)}

\paragraph*{Parameters:} 
\begin{itemize*}
\item[] \verb|RealMatrix| (Column Vector )$\bar{\mu}$, $n \in \{0, 1, 2, \ldots \}$

\item[] \verb|RealMatrix| (Column Vector) $\Sigma$, symmetric and positive semi-definite.

\end{itemize*}

\paragraph*{Support:} \verb|RealMatrix|, $\bar{x} \in \mathbb{Z}^{k}, s.t. x_{i} \in \{0, 1, 2, \ldots, n \} \forall i \in \{0, 1, 2, \ldots, k \}  $ 

\paragraph*{Probability mass function:}
\[
	P(X = x_{1},\ldots,x_{k}) = \frac{n!}{x_{1}! \times \ldots \times x_{k}!} p_{1}^{x_{1}} \ldots p_{k}^{x_{k}}
\]

\paragraph*{Example:}
The following code defines a random function symbol \verb|x| distributed according to a Multivariate Gaussian with mean $\bar{\mu} = 
\begin{pmatrix}
  1  \\
  1  \\ 
 \end{pmatrix}$ and covariance $\Sigma = 
 \begin{pmatrix}
   1 & 0 \\
   0 & 3 \\
  \end{pmatrix}
 $
\begin{blogcode}
fixed RealMatrix mu = [1; 1];
fixed RealMatrix covariance = [1, 0; 0, 3];
random RealMatrix x ~ MultivarGaussian(mu, covariance);
\end{blogcode}

\subsection{NegativeBinomial($r$, $p$)}
NegativeBinomial distribution generates the number of failures until the $r^{\text{th}}$ success in a sequence of independent Bernoulli trials each with probability of success $p$.

\paragraph*{Parameters:} 
\begin{itemize*}
\item[] \verb|Integer| $r$, $r \in \{0, 1, 2, \ldots \}$
\item[] \verb|Real| $p$, $0 \leq p \leq 1$ 
\end{itemize*}

\paragraph*{Support:} \verb|Integer|, $k \in \{0, 1, 2, \ldots \}$ 

\paragraph*{Probability mass function:}
\[
	P(X = k) = \binom{k + r - 1}{k} p^{r} (1-p)^{k} 
\]

\paragraph*{Example:}
The following code defines a random function symbol \verb|x| distributed according to a Negative Binomial distribution with probability of success $p = 0.8$ and number of failures $r = 2$.
\begin{blogcode}
random Integer x ~ NegativeBinomial(2, 0.8);
\end{blogcode}


\subsection{Poisson($\lambda$)}
Given a success probability $p$ very close to $0$, and $n$ independent Bernoulli trials such that $\lambda = np$, the Poisson distribution generates a good approximation to the number of successful trials.

\paragraph*{Parameters:} 
\begin{itemize*}
\item[] \verb|Real| $\lambda$, $\lambda > 0$ 
\end{itemize*}

\paragraph*{Support:} \verb|Integer|, $k \in \{0, 1, 2, \ldots \}$ 

\paragraph*{Probability mass function:}
\[
	P(X = k) = exp(-\lambda) * \frac{\lambda^{k}}{k!}
\]

\paragraph*{Example:}
The following code defines a random function symbol \verb|x| distributed according to a Poisson distribution with mean $\lambda = 3.0$
\begin{blogcode}
random Integer x ~ Poisson(3.0);
\end{blogcode}

\subsection{UniformChoice($S$)}

UniformChoice distribution chooses an element uniformly from the set $S$, whose objects are all of type $T$.

\paragraph*{Parameters:} 
\begin{itemize*}
\item[] \verb|Real| $S$, a set of objects 
\end{itemize*}

\paragraph*{Support:} \verb|T|, $k \in S$ 

\paragraph*{Probability mass function:}
\[
	P(X = k) = \frac{1}{|S|}
\]

\paragraph*{Example:}
The following code defines a random function symbol \verb|x| that selects between the colors Green, Red, and Blue uniformly at random.

\begin{blogcode}
type Color;
distinct Color Green, Red, Blue;
random Color x ~ UniformChoice({c for Color c});
query x;
\end{blogcode}

\subsection{UniformInt($a$, $b$)}
UniformInt distribution generates an \verb|Integer| uniformly at random between the between a lower bound $a$ and an upper bound $b$. $a$ and $b$ are both included in the set of integers which are uniformly sampled from.

\paragraph*{Parameters:} 
\begin{itemize*}
\item[] \verb|Integer| $a$, $a \in \mathbb{Z}$
\item[] \verb|Integer| $a$, $b \in \mathbb{Z}$
\item[] $a \leq b$
 
\end{itemize*}

\paragraph*{Support:} \verb|Integer|, $k \in \{a, a+1, \ldots, b-1, b \}$ 

\paragraph*{Probability mass function:}
\[
	P(X = k) = \left\{
		  \begin{array}{lr}
		    \frac{1}{b - a + 1} & a \leq k \leq b \\
		    0 					& \text{else}
		  \end{array}
		\right.
\]]

\paragraph*{Example:}
The following code defines a random function symbol \verb|x| that selects uniformly at random from the set of integers $\{1, 2, 3\}$

\begin{blogcode}
random Integer x ~ UniformInt(1, 3);
\end{blogcode}

\subsection{UniformReal($a$, $b$)}

UniformReal distribution generates a \verb|Real| uniformly at random between the between a lower bound $a$ and an upper bound $b$. The lower bound $a$ is inclusive but the upper bound is exclusive.

\paragraph*{Parameters:} 
\begin{itemize*}
\item[] \verb|Real| $a$, $a \in \mathbb{R}$
\item[] \verb|Real| $b$, $b \in \mathbb{R}$
\item[] $a < b$
 
\end{itemize*}

\paragraph*{Support:} \verb|Real|, $x \in [a, b)$ 

\paragraph*{Probability density function:}
\[
	f(x) = \left\{
		  \begin{array}{lr}
		    \frac{1}{b - a + 1} & a \leq x < b \\
		    0 					& \text{else}
		  \end{array}
		\right.
\]

\paragraph*{Example:}
The following code defines a random function symbol \verb|x| that generates a \verb|Real| uniformly at random between 0.0 and 1.0.

\begin{blogcode}
random Real x ~ UniformReal(0, 1);
\end{blogcode}

\subsection{UniformVector($v$)}

$v$ is an argument list of $2 \times 1$ \verb|RealMatrix| row vectors with arbitrary length $k$. Denote $v_{i,L}$ as the first element in the $i^{\text{th}}$ row vector and $v_{i, U}$ as the second element in the $i^{\text{th}}$ row vector. UniformVector generates a \verb|RealMatrix| column vector of dimension $k$ where element $i$ is uniformly distributed between $v_{i,L}$ and $v_{i,U}$.

\paragraph*{Parameters:} 
\begin{itemize*}
\item[] \verb|RealMatrix| $v_{1}$, $v_{1} \in \mathbb{R}^{2}, v_{1,L} < v_{1,U}$
\item[] $\ldots$
\item[] \verb|RealMatrix| $v_{k}$, $v_{k} \in \mathbb{R}^{2}, 
v_{k,L} < v_{k,U}$

\end{itemize*}

\paragraph*{Support:} 
$$
\verb|RealMatrix| \text{(Column Vector)}, 
\bar{x} \in \mathbb{R}^{k}, \bigcap_{i=1}^{k} v_{i,L} \leq \bar{x_{i}} \leq v_{i,U}  
$$

\paragraph*{Probability density function:}
\[
	f(\bar{x}) = \frac{1}{\prod_{i=1}^{k} (v_{i,U} - v_{i,L})}
\]

\paragraph*{Example:}
The following code defines a random function symbol \verb|x| that generates a \verb|RealMatrix| column vector of dimension $2$ whose first element is a \verb|Real| that is uniformly distributed between $0$ and $0.5$ and whose second element is a \verb|Real| that is uniformly distributed between $5$ and $10$.

\begin{blogcode}
fixed RealMatrix a = [0, 0.5];
fixed RealMatrix b = [5, 10];
random RealMatrix x ~ UniformVector(a, b);
\end{blogcode}


\begin{table}[H]
\centering
\caption{Distributions in \bl}
\begin{tabular}{ c c c p{2in} }
\toprule 
distribution & argument type & value  & example \\ 
 \midrule
Bernoulli & Real in [0,1] & binary 0/1 & Bernoulli(0.8); \\ 
Beta & Real, Real & Real in [0,1] & Beta(1.0, 1.0); \\ 
Binomial & Integer, Real & Integer & Binomial(10, 0.5); \\ 
BooleanDistrib & Real in [0,1] & Boolean & BooleanDistrib(0.8); \\ 
Categorical & Map & & see main text \\
Dirichlet & Column Vector & Column Vector & 
fixed RealMatrix x = [1; 1; 1];
Dirichlet(x); \\
Discrete & ColumnVector & Integer & fixed RealMatrix x = [2; 3; 5]; Discrete(x);
 \\
Exponential & Real & Real & Exponential(2.0); \\ 
Gamma & Real, Real & Real & Gamma(3, 2.0); \\ 
Gaussian & Real, Real & Real & Gaussian(2.0, 1.0); \\ 
Geometric & Real in [0,1] & nonnegative Integer & Geometric(0.5); \\ 
Laplace & Real, and positive Real & Real & Laplace(0, 1.0) \\ 
Multinomial & Integer, ColumnVector & ColumnVector & 
fixed RealMatrix x = [1; 1; 2];
Multinomial(5, x); \\
MultivarGaussian & ColumnVector, MatrixReal & ColumnVector & 	fixed RealMatrix mu = [1; 1];
	fixed RealMatrix cov = [1, 0; 0, 3];
	MultivarGaussian(mu, cov); \\
NegativeBinomial & Integer, Real in [0,1] & Integer & NegativeBinomial(4, 0.5); \\ 
Poisson & Real & nonnegative Integer & Poisson(6.0); \\ 
UniformChoice & Set & Element in Set & UniformChoice({p for Person p}) \\
UniformInt & Integer, Integer & Integer & UniformInt(0, 10); \\
UniformReal & Real, Real & Real & UniformReal(0, 1.0); \\
UniformVector & MatrixReal's & RealMatrix &
 fixed RealMatrix a = [0, 0.5];
 fixed RealMatrix b = [5, 10];
 UniformVector(a, b); \\
 \bottomrule
\end{tabular} 
\end{table}