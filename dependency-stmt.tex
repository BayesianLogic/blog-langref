A {\em dependency statement} constrains the value of functions applied to objects. In the case of ordinary programming languages,
as well as in the case of built-in functions in \bl, the value of a function applied to a given tuple of input objects is {\em fixed},
i.e., it is the same in all worlds. Dependency statements for fixed functions are described in \secref{fixed-section}. On the other hand, a {\em random} function
is one about whose values there is uncertainty, so that the values may vary across possible worlds. Dependency statements for random functions are described in \secref{random-function-section}.


%%%%%%%%%%%%%%%%%%%%%%%%%%%%%%%%%%%%%%%%%%%%%%%%%%%%%%%%%%%%%%%%%%%%%%%%%%%%%%

\subsection{Fixed functions}\label{fixed-section}

\bl allows the user to declare that there is {\em no} uncertainty concerning the value of a given function; that is, the
value of the function applied to given arguments is the same in all worlds. Such a fixed function is declared as follows:
\begin{blogcode}
fixed type_name0 function_name(type_name1 var1, ...) = 
  fixed_expression;
\end{blogcode}
This statement defines a fixed function with name \texttt{function\_name} whose arguments are {\tt var1} (of type \verb|type_name1|), etc.,
and whose return type is \verb|type_name0|. The logical variables {\tt var1}, {\tt var2}, etc., are implicitly universally quantified over
all elements of the corresponding types.

The function body is a fixed expression, which may be
\begin{itemize*}
\item a literal from one of the built-in types or a declared distinct symbol;
\item a logical variable from the argument list;
\item a \hyperref[builtin-operator-appendix]{built-in operator}, fixed function, or \hyperref[external-function-appendix]{externally defined function} applied to fixed expressions.
\end{itemize*}
Put another way, a fixed expression is one containing no random function applications.

The following example defines a function to calculate the sum of squares:
\begin{blogcode}
random Real sumsquare(Real x, Real y) = x^2 + y^2;
\end{blogcode}

When a function has zero arguments, the resulting empty parentheses may be dropped
in both the declaration statement and in occurrences within expressions.
A fixed function with no arguments is called a \emph{fixed constant}; for
example:
\begin{blogcode}
fixed Real pi = 3.14159;
fixed Real CircleArea(Real r) = pi * r^2;
\end{blogcode}


%%%%
\subsubsection{List interpretations}\label{sec:list-interp}

The \verb|ListInterp| construct allows defining a relation by exhaustively
listing the tuples for which it holds. The first argument of \verb|ListInterp|
is the number of arguments to the relation. The remaining arguments list the
tuples for which the relation is true. The relation is implicitly false for any
tuple not listed. For example:

\begin{blogcode}
fixed Boolean Teaches(Professor p, Course c)
    = ListInterp(2, Smith, CS106,
                    Jones, CS106,
                    Moriarty, Phil80,
                    Jones, Stat10);
\end{blogcode}

With this definition, {\tt Teaches(Jones, CS106)} evaluates to \verb|true|,
while {\tt Teaches(Jones, Math101)} evaluates to \verb|false|.


%%%%
\subsubsection{Tabular interpretations}\label{sec:tabular-interp}

The \verb|TabularInterp| construct allows defining a function by exhaustively
listing the mapping it performs from arguments to return value. The first
argument to \verb|TabularInterp| is the number of arguments to the function.
The remaining arguments list the arguments and return value for which the
function is defined. The function will return \verb|null| for any other tuple
of arguments. For example:

\begin{blogcode}
fixed Integer Score(Student s, Course c)
  = TabularInterp(2, Alice, CS106, 80,
                     Alice, Phil80, 55,
                     Bob, CS106, 65,
                     Bob, Stat10, 35);
\end{blogcode}

With this definition, {\tt Score(Alice, Phil80)} evaluates to \verb|55|, while
{\tt Score(Bob, Phil80)} evalutes to \verb|null|.


%%%%%%%%%%%%%%%%%%%%%%%%%%%%%%%%%%%%%%%%%%%%%%%%%%%%%%%%%%%%%%%%%%%%%%%%%%%%%%

\subsection{Random functions}\label{random-function-section}

To declare a random function, there are two possible forms. The first describes a probabilistic
conditional dependency:
\begin{blogcode}
random type_name0 function_name(type_name1 var1, ...) ~ 
  distribution_expression;
\end{blogcode}
This form defines a random function with name \texttt{function\_name} whose arguments are {\tt var1} (of type \verb|type_name1|), etc.,
and whose return type is \verb|type_name0|. The logical variables {\tt var1}, {\tt var2}, etc., are implicitly universally quantified over
all elements of the corresponding types and may appear in the distribution expression. The statement asserts that for any possible instantiation of the
logical variables with objects, the resulting random variable has a conditional probability distribution
described by the corresponding instantiation of the {\tt distribution\_expression}. The {\tt distribution\_expression} must return a concrete distribution. 

For example, the following statement says the height of a tree has a Gaussian distribution
whose mean depends linearly on the tree's age and the growth rate of its species:
\begin{blogcode}
random Real height(Tree x) ~
    Gaussian(Growthrate(Species(x))*Age(x), 4.0);
\end{blogcode}
The full syntax of distribution expressions is described in \secref{distribution-section}.

The second form describes a deterministic dependency:
\begin{blogcode}
random type_name0 function-name(type_name1 var1, ...) = expression;
\end{blogcode}
As before, the logical variables may appear in the expression.
Such a declaration is distinct from a fixed function declaration because, although the dependency is deterministic,
the expression may contain other random function symbols. For example, 
to express the fact that the observed value {\tt Y(t)} of some temporal process 
is the underlying state {\tt X(t)} plus an additive Gaussian noise term, one may write
\begin{blogcode}
random Real Epsilon(Timestep t) ~ Gaussian(0.0,1.0);
random Real Y(Timestep t) = X(t) + Epsilon(t);
\end{blogcode}
